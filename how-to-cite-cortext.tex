\documentclass{article}
\usepackage{graphicx}
\usepackage{hyperref}
\usepackage{fancyvrb}
\usepackage[backend=biber,datamodel=software]{biblatex}
\usepackage{software-biblatex}

\title{
  How To Cite CorText Manager
}

\author{
  Joenio Marques da Costa \\
  Université Gustave Eiffel
}

\date{\today}

\addbibresource{how-to-cite-cortext.bib}
\addbibresource{biblatex-software.bib}
\addbibresource{bibtex.bib}
\ExecuteBibliographyOptions{
  halid=true,
  swhid=true,
  swlabels=true,
  vcs=true,
  license=true}

\begin{document}

\maketitle

\section{Introduction}

Software desempenha um pilar fundamental na ciencia moderna em todos os campos
do conhecimento, entretanto nao é ainda reconhecido como um produto de
primeira classe, algo necessario para gerar credito academico e incentivar
pesquisadores e organizacoes na construcao de software para a
ciencia \cite{alliez_attributing_2020}.

Este panorama tem sido alvo de debates, como por exemplo,
como aplicar os principios FAIR (Findable,
Accessible, Interoperable e Reusable) a software \cite{noauthor_fair_nodate},
e propostas como por exemplo o Citation
File Format (CFF) \cite{chue_hong_software_2019}, um
formato YAML para descricao de metadados de software, ou
CodeMeta \cite{druskat_citation_2019}, um vocabulario conceitual em
formato JSON-LD para interoperabilidade de repositorios de software.

Muitas destas propostas sao construidas a partir de debates centrados nos principios para citacao de software \cite{smith_software_2016}, resumidos a seguir:

\begin{description}
  \item[Importancia] Software deve ser considerado um produto pesquisa legitimo e citavel;
  \item[Credito e atribuicao] Citacao a software deve gerar credito e reconhecimento academico;
  \item[Identificador unico] A citacao ao software deve incluir uma maneira unica de identificacao e interoperaval;
  \item[Persistencia, persistente] O identificador unico e os metadados do software devem ser persistir;
  \item[Acessibilidade] Citacoes ao software devem facilitar acesso ao software e ao seus metadados;
  \item[Especificidade] Citacao a software deve facilitar acesso e identificacao a versoes espeficicas do software.
\end{description}

Tomando estes principios como ponto de partida e as propostas de formatos de
arquivos e ferramentas propostas neste debate este documento resume uma
proposta de como citar o software CorText Manager apropriadamente em trabalhos
academicos, buscando assim incentivar credito e atribuicao as instituicoes,
pesquisadores, engenheiros e demais atores envolvidos na construcao e
financiamento da plataforma CorText Manager.

\section{Cite CorText Manager with Bib\LaTeX} \label{example-biblatex}

Abaixo segue um exemplo de como o CorText Manager ao ser citado\cite{cortext_manager_v2} é visualizado nas referencias
bibliograficas, este exemplo usa como base os
dados descritos no arquivo {\em biblatex-software.bib} e é modo recomendavel ao citar CorText Manager.

\printbibliography[type=software,title={\small References (example)}]

\subsection{Bib\LaTeX \ file}

Below you see the content of the Bib\LaTeX \ file
\href{https://github.com/cortext/how-to-cite-cortext/blob/main/biblatex-software.bib}{\em biblatex-software.bib}
with the Bib\LaTeX \ Software extension\cite{softwareheritageorg_citing_2020}.

\VerbatimInput[fontsize=\small]{biblatex-software.bib}

When using this file you can see the result of reference format citing CorText
Manager \cite{cortext_manager_v2} on the references section of this document.

\section{Cite CorText Manager with Bib\TeX} \label{example-bibtex}

\printbibliography[type=misc,keyword=cortext,title={\small References (example)}]

\subsection{Bib\TeX \ file}

Below you see the content of the Bib\TeX \ file
\href{https://github.com/cortext/how-to-cite-cortext/blob/main/bibtex.bib}{\em bibtex.bib},
this file is created from
\href{https://github.com/cortext/how-to-cite-cortext/blob/main/CITATION.cff}{\em CITATION.cff}
converted by {\em cffconvert} tool.

\VerbatimInput[fontsize=\small]{bibtex.bib}

When using this file you can see the result of reference format citing CorText
Manager \cite{cortext_manager_v2_bibtex} on the references section of this document.

\section{CorText Manager metadata}

The CorText Manager metadata for academic citation is being maintained mainly in two files:

\begin{description}
  \item[CITATION.cff] Citation File Format (CFF) file with metadata about the software CorText Manager,
    this file is the main file to centralize all metadata required for citation.
  \item[biblatex-software.bib] BibLaTeX file with Software extension,
    this file is a transcription of all informations from the {\em CITATION.cff} file.
\end{description}

The example on Section \ref{example-biblatex} was generated by using the file
{\em biblatex-software.bib}, besides those two files there are some other
files that must be described:

\begin{description}
  \item[codemeta.json] CodeMeta file generated from {\em CITATION.cff} by the
    {\em cffconvert} tool, this file is a machine-readable file in a
    interchangeable JSON-LD format.
  \item[bibtex.bib] BibTeX file generated from {\em CITATION.cff} by the {\em
    cffconvert} tool, this file is useful for whom is using BibTeX as referencing
    system.
\end{description}

The example on Section \ref{example-bibtex} was generated by using the {\em
bibtex.bib} file, instructions on how to update those files can be found at
\href{https://github.com/cortext/how-to-cite-cortext/blob/main/AUTHORS.md}{\em AUTHORS.md}
document.

%% * criar um espaco no site www.cortext.net e documentar como citar, com links e referencias, exemplos, etc
%% * OK criar arquivo `CITATION.cff` com metadados do CorText, autores, abstract, url, etc
%%   * https://citation-file-format.github.io/cff-initializer-javascript cliente web para gerar arquivo inicial
%% * OK gerar arquivo `codemeta.json` a partir do `CITATION.cff`, outros formatos sao suportados tambem
%%   * https://github.com/citation-file-format/cff-converter-python
%%   * `cffconvert -i CITATION.cff -f codemeta -o codemeta.json`
%% * OK publicar `CITATION.cff` e `codemeta.json`
%% * OK entender e documentar qual utilidade do codemeta.json
%%   * O codemeta.json serve para interoperibilidade de sistemas integrando a ciencia em metadados de software, comunicacao entre github e figshare por ex
%% * OK melhorar o bibtex.bib geradi pelo cffconverter pois eh mto pobre n preserva quase nada do CITATION.cff
%%   * nao vale a pena melhorar, o core do cffconverter gera um simples bibtex e eh isso, manter so como exemplo mas preferir o biblatex-software
%% * OK olhar e escrever bibtex CorText para o bibtex-software desenvolvido pelo pessoal do software heritage, criar arquivo `biblatex-software.bib`
%% * OK criar RESEARCHERS.md (ou USERS.md) e DEVELOPERS.md para documentar instrucoes para quem precisa citar CorText e para os mantenedores do CorText
%% * OK HAL nao documenta nada sobre aceitar software metadata, a descricao dos campos ao depositor n mostra nada onde software se encaixe
%%   * creio que a melhor forma de arquivar os metadados do software eh atraves de um software paper, ele pode ser publicado no HAL
%% * OK criar um persistent ID para o CorText, opcoes sao DOI com um software paper ou um SWHID (vou tentar o SWHID primeiro)
%%   * usar URL do gitlab ou github para este repositorio, exemplo: https://github.com/cortext/how-to-cite-cortext

\printbibliography

\end{document}
