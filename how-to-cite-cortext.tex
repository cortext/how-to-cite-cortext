\documentclass{article}
\usepackage{graphicx}
\usepackage{hyperref}
\usepackage{fancyvrb}
\usepackage[backend=biber,datamodel=software]{biblatex}
\usepackage{software-biblatex}
\usepackage{minted}
\usemintedstyle{algol}

\title{
  How To Cite CorTexT Manager
}

\author{
  Marques da Costa, Joenio\\
  Université Gustave Eiffel
  \and
  Barbier, Marc\\
  LISIS - Laboratoire Interdisciplinaire Sciences, Innovations, Sociétés
  \and
  Villard, Lionel\\
  ESIEE Paris
  \and
  Daniel, Luis\\
  LISIS - Laboratoire Interdisciplinaire Sciences, Innovations, Sociétés
}

\date{\today}

\addbibresource{how-to-cite-cortext.bib}
\addbibresource{biblatex-software.bib}
\addbibresource{bibtex.bib}
\ExecuteBibliographyOptions{
  halid=true,
  swhid=true,
  swlabels=true,
  vcs=true,
  license=true}

\begin{document}

\maketitle

\section{Introduction}

Software and Platform availability play a fundamental role in scientific
activities and front of knowledge development. It concerns many disciplines,
interdisciplinary research and open-sciences. Software, services, datasets and
IT assets are more and more being assembled to give birth to research
infrastructures.  It has become necessary to increase the impact and radical
empowerment of those applications and infrastructures on science dynamics, and
thus to recognize Software and Infrastructure for research as a first class
product. Researchers, engineers and their organizations must be encouraged by
academic credit, so that the value of their involvement in support and
development of software and platforms for science is acknowledged
\cite{alliez_attributing_2020}.

The European policy of research infrastructure plays a key role in this context. It intends to stimulate and foster the professionalization of infrastructure technology and management, in the interest of ensuring the sustainability of software and infrastructures on the service of science. On that account, the engagement of scientists and engineers in software and platform archictecture needs to be recongnized as a
key contribution to ongoing and future dynamics of Science and Innovation in
Societies.

One key issue in this topic is the role of software as key assets for new
front of knowledge. This point has been and -still is- a matter of debate,
such as how to apply the FAIR (Findable, Accessible, Interoperable and
Reusable) principles to software \cite{noauthor_fair_nodate}, and proposals
such as the Citation File Format (CFF) \cite{chue_hong_software_2019}, a YAML
format for describing software metadata, or CodeMeta
\cite{druskat_citation_2019}, a JSON-LD conceptual vocabulary for software
repositories interoperability.

Many of these proposals are built around debates centered on the software
citation principles \cite{smith_software_2016}, which are summarized below:

\begin{description}
  \item[Importance:] Software should be considered a legitimate and citable product of research.
  \item[Credit and attribution:] Software citations should facilitate giving scholarly credit and recognizing.
  \item[Unique identification:] A software citation should include a globally unique identification.
  \item[Persistence:] Unique identifiers and metadata describing the software should persist.
  \item[Accessibility:] Software citations should facilitate access to the software itself and to its metadata.
  \item[Specificity:] Software citations should facilitate identification of the specific version of software that was used.
\end{description}

Based on these principles and considering some of the more recent ideas on how
to cite software, here follows a proposal of how the CorText Manager as a on
line application or software should be cited in academic papers.

\section{Cite CorText Manager with Bib\LaTeX} \label{example-biblatex}

Within the general issue exposed in section 1, this note intends to specify the
manner in which CorText Manager should be cited when it is used as a software
that is developed and delivered by the CorTexT team as an online application
for scientific work.

When cited the format of citation is to be clearly defined as shown hereunder
\cite{cortext_manager_v2} is displayed in the bibliographic references. This
example uses the data described in the file {\em biblatex-software.bib}.  This
format is the one recommended to cite CorText Manager in an appropriate way.

\printbibliography[type=software,title={\small References to Software (example)}]

\subsection{Bib\LaTeX \ file}

Below is the content of the Bib\LaTeX \ file
\href{https://github.com/cortext/how-to-cite-cortext/blob/main/biblatex-software.bib}{\em biblatex-software.bib}
with the Bib\LaTeX \ Software extension\cite{softwareheritageorg_citing_2020}.

\inputminted[fontsize=\small, breaklines=true, breaksymbolleft=]{bib}{biblatex-software.bib}

When using this file it is possible to see the result of reference format citing CorText
Manager \cite{cortext_manager_v2} on the references section of this document.

\section{Cite CorText Manager with Bib\TeX} \label{example-bibtex}

\printbibliography[type=misc,keyword=cortext,title={\small References (example)}]

\subsection{Bib\TeX \ file}

Below is the content of the Bib\TeX \ file
\href{https://github.com/cortext/how-to-cite-cortext/blob/main/bibtex.bib}{\em bibtex.bib},
this file is created from
\href{https://github.com/cortext/how-to-cite-cortext/blob/main/CITATION.cff}{\em CITATION.cff}
converted by {\em cffconvert} tool.

\inputminted[fontsize=\small, breaklines=true, breaksymbolleft=]{bib}{bibtex.bib}

When using this file it is possible to see the result of reference format citing CorText
Manager \cite{cortext_manager_v2_bibtex} on the references section at the end of this document.

\subsection{APA file}

Below is the content of the APA file
\href{https://github.com/cortext/how-to-cite-cortext/blob/main/apalike.apa}{\em apalike.apa},
this file is created from
\href{https://github.com/cortext/how-to-cite-cortext/blob/main/CITATION.cff}{\em CITATION.cff}
converted by {\em cffconvert} tool.

\inputminted[fontsize=\small, breaklines=true, breaksymbolleft=]{text}{apalike.apa}

\section{CorText Manager metadata}

The CorText Manager metadata for academic citation is being maintained mainly in the
{\em CITATION.cff} file.

\begin{description}
  \item[CITATION.cff] Citation File Format (CFF) file with metadata about the software CorText Manager,
    this file is the main file to centralize all metadata required for citation.
\end{description}

The {\em CITATION.cff} file can be read by the {\em cffconvert} tool and
translated in some other formats, like {\em codemeta.json}, {\em bibtex.bib}
and {\em apalike.apa}.

\begin{description}
  \item[codemeta.json] CodeMeta file generated from {\em CITATION.cff} by the
    {\em cffconvert} tool, this file is a machine-readable file in a
    interchangeable JSON-LD format.
  \item[bibtex.bib] BibTeX file generated from {\em CITATION.cff} by the {\em
    cffconvert} tool, this file is useful for whom is using Bib\TeX \ as referencing
    system.
  \item[apalike.apa] APA file generated from {\em CITATION.cff} by the {\em
    cffconvert} tool, this file is a plaintext format.
\end{description}

%The example on Section \ref{example-bibtex} was generated by using the {\em
%bibtex.bib} file, instructions on how to update those files can be found at
%\href{https://github.com/cortext/how-to-cite-cortext/blob/main/AUTHORS.md}{\em AUTHORS.md}
%document.

Besides those files there also the {\em biblatex-software.bib}

%The example on Section \ref{example-biblatex} was generated by using the file
%{\em biblatex-software.bib}, besides those two files there are some other
%files that must be described:

\begin{description}
  \item[biblatex-software.bib] Bib\LaTeX \ file with Software extension,
    this file is a transcription of all informations from the {\em CITATION.cff} file.
\end{description}

The {\em biblatex-software.bib} is the preferred way for citing CorTexT Manager
as it offers much more rich metadata then the {\em bibtex.bib}

\printbibliography

\end{document}
