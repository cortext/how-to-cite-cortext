\documentclass{article}
\usepackage{graphicx}
\usepackage{hyperref}
\usepackage{fancyvrb}

\usepackage[backend=biber,datamodel=software]{biblatex}
\usepackage{software-biblatex}

\title{
  How To Cite CorText Manager
}

\author{
  Joenio Marques da Costa \\
  Université Gustave Eiffel
}

\date{\today}

\addbibresource{how-to-cite-cortext.bib}
\addbibresource{biblatex-software.bib}
\addbibresource{bibtex.bib}
\ExecuteBibliographyOptions{
  halid=true,
  swhid=true,
  swlabels=true,
  vcs=true,
  license=true}

\begin{document}

\maketitle

\begin{abstract}
Short introduction to subject of the paper \ldots 
\end{abstract}

\section{Introduction}

Software desempenha um pilar fundamental para a ciencia moderna em todos os
campos de pesquisa, entretanto ainda ha carencia de uma maior adocao de
praticas para citar e tornar software um produto de primeira classe na ciencia,
o reconhecimento do software como objeto legitimo no campo academico eh
importante para incentivar pesquisadores e organizacoes construindo software
para a pesquisa bem como para preservar o conhecimento presente nestes projetos
\cite{alliez_attributing_2020}.

A comunidade cientifica vem debatendo propostas de como citar software apropriadamente
e como dar visibilidade aos pesauisadores desenvolvedores e aos projetos em si,
neste contexto tem surgido debates sobre FAIR (Findable, Accessible,
Interoperable e Reusable) aplicado a software\cite{noauthor_fair_nodate}., formatos de arquivos Citation File Format (CFF),
o Citation File Format (CFF) eh um arquivo no formato YAML 1.2 legivel para humanos e
maquinas que prove metadata para citacao de software \cite{chue_hong_software_2019},
formatos JSON para interoperabilidade,
O formato CodeMeta.json captura metadados de software, e eh entendido como um crescente
numero de repositorio digital, eh tambem facil de converter este para muitos esquemas
comuns de metadados usado por repostorios de software,
the goal of CodeMeta is to create a concept vocabulary that can be used to
standardize the exchange of software metadata across repositories and
organizations (GitHub, figshare, Zenodo) \cite{druskat_citation_2019}.

Todo este debate tem sido delineado por alguns principios ao citar software sao resumidos em: Importancia, software deve ser considerado um produto pesquisa legitimo e citavel;
Credito e atribuicao, citacao a software deve gerar credito e reconhecimento academico;
Identificador unico, a citacao ao software deve incluir uma maneira unica de identificacao e interoperaval;
Persistencia, persistente, o identificador unico e os metadados do software devem ser persistir;
Acessibilidade, citacoes ao software devem facilitar acesso ao software e ao seus metadados;
Especificidade, citacao a software deve facilitar acesso e identificacao a versoes espeficicas do software \cite{smith_software_2016}

%O arquivo CITATION.cff file tem se tornado amplamento reconhecido entre as
%diversas iniciativas e possui integracao e ferramentas para versao entre outros
%formatos, CodeMeta, BibTex etc \cite{saragon_standard_2017}.

Assim, este texto resume uma proposta de como citar o software CorText Manager
apropriadamente em trabalhos academicos contribuindo para aumentar o
reconhecimento tanto do software, quanto dos pesquisadores, desenvolvedores e
demais colaboradores envolvidos na construcao e financiamento desta plataforma
que tem sido constantemente desenvolvida ao longo de 10 anos. ...

\section{CorText Manager metadata}

The CorText Manager metadata for academic citation is being maintaned in two files:

\begin{description}
  \item[CITATION.cff] Citation File Format (CFF) fila with data about the software CorText Manager,
    this file is the main file to concentrate CorText Manager metadata.
  \item[biblatex-software.bib] BibLatex file with extension for Software for CorText Manager,
    this file is a transcription of informations from the {\em CITATION.cff} file.
\end{description}

Besides those two main files there are also some others files important to be described as follows:

\begin{description}
  \item[codemeta.json] CodeMeta file generated from `CITATION.cff` by the {\em cffconvert} tool,
    this file is machine-readable file in a interchangeable format.
  \item[bibtex.bib] BibTex file generated from `CITATION.cff` by the {\em cffconvert} tool,
    this file is useful for whom is writing paper with LaTeX and is using BibTex as referencing
    system.
\end{description}

\section{Citing CorText with biblatex-software}

Below you see the content of the {\em biblatex-software.bib}\footnote{Available
at \url{https://github.com/cortext/how-to-cite-cortext}} in BibLaTeX format
with software style extension\cite{softwareheritageorg_citing_2020}.

\VerbatimInput[fontsize=\small]{biblatex-software.bib}

%% * criar um espaco no site www.cortext.net e documentar como citar, com links e referencias, exemplos, etc
%% * OK criar arquivo `CITATION.cff` com metadados do CorText, autores, abstract, url, etc
%%   * https://citation-file-format.github.io/cff-initializer-javascript cliente web para gerar arquivo inicial
%% * OK gerar arquivo `codemeta.json` a partir do `CITATION.cff`, outros formatos sao suportados tambem
%%   * https://github.com/citation-file-format/cff-converter-python
%%   * `cffconvert -i CITATION.cff -f codemeta -o codemeta.json`
%% * OK publicar `CITATION.cff` e `codemeta.json`
%% * OK entender e documentar qual utilidade do codemeta.json
%%   * O codemeta.json serve para interoperibilidade de sistemas integrando a ciencia em metadados de software, comunicacao entre github e figshare por ex
%% * OK melhorar o bibtex.bib geradi pelo cffconverter pois eh mto pobre n preserva quase nada do CITATION.cff
%%   * nao vale a pena melhorar, o core do cffconverter gera um simples bibtex e eh isso, manter so como exemplo mas preferir o biblatex-software
%% * OK olhar e escrever bibtex CorText para o bibtex-software desenvolvido pelo pessoal do software heritage, criar arquivo `biblatex-software.bib`
%% * criar RESEARCHERS.md (ou USERS.md) e DEVELOPERS.md para documentar instrucoes para quem precisa citar CorText e para os mantenedores do CorText
%% * HAL nao documenta nada sobre aceitar software metadata, a descricao dos campos ao depositor n mostra nada onde software se encaixe
%%   * creio que a melhor forma de arquivar os metadados do software eh atraves de um software paper, ele pode ser publicado no HAL
%% * OK criar um persistent ID para o CorText, opcoes sao DOI com um software paper ou um SWHID (vou tentar o SWHID primeiro)
%%   * usar URL do gitlab ou github para este repositorio, exemplo: https://github.com/cortext/how-to-cite-cortext

When using this file you can see the result of reference format citing CorText
Manager \cite{cortext_manager_v2} on the references section of this document.

\section{Citing CorText with bibtex}

Below you see the content of the {\em bibtex.bib}\footnote{Available
at \url{https://github.com/cortext/how-to-cite-cortext}} in BibTeX format,
this file is created from {\em CITATION.cff}\footnote{Available
at \url{https://github.com/cortext/how-to-cite-cortext}} converted by the {\em cffconvert} tool.

\VerbatimInput[fontsize=\small]{bibtex.bib}

When using this file you can see the result of reference format citing CorText
Manager \cite{cortext_manager_v2_bibtex} on the references section of this document.

\printbibliography

\end{document}
