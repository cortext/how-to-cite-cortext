\documentclass{article}
\usepackage{graphicx}
\usepackage{hyperref}

\usepackage[backend=biber,datamodel=software]{biblatex}
\usepackage{software-biblatex}

\title{
  How To Cite CorText Manager
}

\author{
  Joenio Marques da Costa \\
  Université Gustave Eiffel
}

\date{\today}

\addbibresource{how-to-cite-cortext.bib}
\addbibresource{biblatex-software.bib}
\ExecuteBibliographyOptions{
  halid=true,
  swhid=true,
  swlabels=true,
  vcs=true,
  license=true}

\begin{document}

\maketitle

\section{Introduction} % motivation

Guia simplificado passo a passo para apoiar autores de software a fazer o software ser FAIR.
Os principios FAIR surgiram de conceitos do area de gerenciamento de dados, o acronimo significa
Findable, Accessible, Interoperable e Reusable.
Sao uma bandeira para promover boas praticas em gerenciamento de dados, mas nao sao aplicavei diretamente a software
devido a natureza diferente dos dois objetos, dados e software sao de longe dois artefatoss muito diferentes, apesar de
em certas circustancias serem tratados iguais.

Ajuda a publicar em repositorios publicos e controle de versao.
Indica selecionar uma licenca e da apoio e informacao de como escolher.
Indica registrar o codigo numa indice registro de alguma comunidade,
indica diversos servicos, por pais, por dominio, por plataforma, etc.
Crie as condicoes para o software ser citavel.
Este eh o ponto do meu interesse para definir como Cortext deve ser citado,
aqui parece ser a referencia mais fresca e atualizada,
a recomendacao eh usar CodeMeta e Citation File Format CFF,
era o q eu ja vinha sentindo q estes eram mesmo as ultimas recomendacoes sobre o tema.
E por fim recomenda boas praticas de qualidade, testes, documentacao etc,
nao se aplica para o Cortext pois o cortext ja esta estabelecido nesse contexto
\cite{noauthor_fair_nodate}.



Ha uma infinidade de ferramentas para ler, gerar ou converter CFF de e from para outros formatos, etc.
Esta pagina tem recomendacoes como usar CITATION.cff file, como gerar DOI e o que fazer qdo o software eh proprietario,
neste caso n eh possivel gerar doi pois nao se ha o repositorio do projeto, entao ha recomendacoes de como
citar e como dar instrucoes para que outros citem o projeto, exatamente o que necessito para definir corretamente
para o Cortext, devo ler com atencao e tomar notas etalhadas deste texto aqui neste documento.
TODO ler e tomar notas
\cite{saragon_standard_2017}.


CodeMeta deve ser gerar um arquivo codemeta.json e colocar na raiz do repositorio Git, Github, ok mas quais as
diferencas, similaridades, usar um ou outro em relacao ao arquivo CITATION.cff? Qual usar?
Este artigo tem todas as respostas, devo ler urgente e com atencao, eh o ponto fundamental de como definir o arquivo e como
se relacionado com codemeta etc...
nesse link eh possivel encontrar as versoes e links para doc de cada especificacao https://citation-file-format.github.io/about/
\cite{druskat_citation_2019}.


Este documento oferece um resumido, generico checklist para desenvolvedores de software
(seja software livre ou proprietario) usado em pesquisas podem usar para garantir que eles
estao seguindo boas praticas sobre citacao de software. Isto ira ajudar desenvolvedores de
obter credito para o software que eles criam, e melhorara transparencia, reproducibilidade
e reuso.
Quais sao os formatos de metadados apropriados?
O formato CodeMeta.json [3]  captura metadados de software, e eh entendido como um crescente
numero de repositorio digital, eh tambem facil de converter este para muitos esquemas
comuns de metadados usado por repostorios de software.
O Citation File Format (CFF) [4] eh um arquivo no formato YAML 1.2 legivel para humanos e
maquinas que prove metadata para citacao de software.
\cite{chue_hong_software_2019}.

Referencia sobre software-bibtex \cite{softwareheritageorg_citing_2020}.

The goal of CodeMeta is to create a concept vocabulary that can be used to standardize the exchange of software metadata across repositories and organizations
With codemeta, we want to formalize the schema used to map between the different services (GitHub, figshare, Zenodo) to help others plug into existing systems. Having a standard software metadata interoperability schema will allow other data archivers and libraries join in. This will help keep science on the web shareable and interoperable!

\section{Conclusions}

* criar um espaco no site www.cortext.net e documentar como citar, com links e referencias, exemplos, etc
* OK criar arquivo `CITATION.cff` com metadados do CorText, autores, abstract, url, etc
  * https://citation-file-format.github.io/cff-initializer-javascript cliente web para gerar arquivo inicial
* OK gerar arquivo `codemeta.json` a partir do `CITATION.cff`, outros formatos sao suportados tambem
  * https://github.com/citation-file-format/cff-converter-python
  * `cffconvert -i CITATION.cff -f codemeta -o codemeta.json`
* OK publicar `CITATION.cff` e `codemeta.json`
* OK entender e documentar qual utilidade do codemeta.json
  * O codemeta.json serve para interoperibilidade de sistemas integrando a ciencia em metadados de software, comunicacao entre github e figshare por ex
* OK melhorar o bibtex.bib geradi pelo cffconverter pois eh mto pobre n preserva quase nada do CITATION.cff
  * nao vale a pena melhorar, o core do cffconverter gera um simples bibtex e eh isso, manter so como exemplo mas preferir o biblatex-software
* OK olhar e escrever bibtex CorText para o bibtex-software desenvolvido pelo pessoal do software heritage, criar arquivo `biblatex-software.bib`
* criar RESEARCHERS.md (ou USERS.md) e DEVELOPERS.md para documentar instrucoes para quem precisa citar CorText e para os mantenedores do CorText
* HAL nao documenta nada sobre aceitar software metadata, a descricao dos campos ao depositor n mostra nada onde software se encaixe
  * creio que a melhor forma de arquivar os metadados do software eh atraves de um software paper, ele pode ser publicado no HAL

O cffconverter ainda nao suporta gerar saida compativel com software-bibtex https://github.com/citation-file-format/cff-converter-python/issues/152,
entao a solucao por agora sera escrever manualmente um arquivo Bibtex manualmente seguindo a extensao software-bibtex.

O biblatex-software nao esta empacotado no debian, um modo de instalar eh com o tlmgr, gerenciador
de pacotes do TexLive.

% $ tlmgr install biblatex-software
% (running on Debian, switching to user mode!)
% (see /usr/share/doc/texlive-base/README.tlmgr-on-Debian.md)
% TLPDB: not a directory, not loading: /home/joenio/texmf
% tlmgr: user mode not initialized, please read the documentation!

O documento /usr/share/doc/texlive-base/README.tlmgr-on-Debian.md desaconselha o uso de tlmgr em usermode, desisti e fiz download
dos arquivos e adicionei aqui no repositorio local. CORRECAO o pacote texlive-bibtex-extra 2021.20211217-1 no Debian testing incluid o biblatex-software.

Dai vejo que meu workflow usual em Latex eh com o bibtex e nao o biblatex, para o biblatex me falta instalar os pacotes abaixo:

* biber
* texlive-bibtex-extra

Baixei os arquivos da versao 1.2-3 em https://www.ctan.org/tex-archive/macros/latex/contrib/biblatex-contrib/biblatex-software
e criei um diretorio local texmf-dist/biblatex-software e crei links simbolicos...

Citando CorText Manager \cite{cortext_manager_v2} ...

%\ifbacktracker
%\printbibheading[title=References \emph{(with backref enabled)}]
%\else
%\printbibheading[title=References \emph{(default style)}]
%\fi
%\printbibliography[heading=subbibliography,type=software,title={Software Projects}]
%\printbibliography[heading=subbibliography,nottype=software,title={Software versions, modules, excerpts and manuals}]
\printbibliography

\end{document}
