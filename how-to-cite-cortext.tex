\documentclass{article}
\usepackage{graphicx}
\usepackage{hyperref}
\usepackage{fancyvrb}
\usepackage[backend=biber,datamodel=software]{biblatex}
\usepackage{software-biblatex}

\title{
  How To Cite CorText Manager
}

\author{
  Joenio Marques da Costa\\
  Université Gustave Eiffel
  \and
  Barbier, Marc\\
  LISIS - Laboratoire Interdisciplinaire Sciences, Innovations, Sociétés
}

\date{\today}

\addbibresource{how-to-cite-cortext.bib}
\addbibresource{biblatex-software.bib}
\addbibresource{bibtex.bib}
\ExecuteBibliographyOptions{
  halid=true,
  swhid=true,
  swlabels=true,
  vcs=true,
  license=true}

\begin{document}

\maketitle

\section{Introduction}

Software and Platform availability play a fundamental role in scientific
activities and front of knowledge development. It concerns many disciplines,
interdisciplinary research and open-sciences. Software, services, datasets and
IT assets are more and more assembled in research infrastructures. There is
need to up-grade the impact and radical empowerment of those application and
infrastructure on science dynamics, and thus to recognize Software and
Infrastructure for research as a first class product.  This means to generate
academic credit and encourage researchers, engineers and their organizations to
be more involved in the support and recognition of software and platforms for
science \cite{alliez_attributing_2020}.

The European policy of research infrastructure plays a key-role and tend to
accelerate this development and also foster the professionalization  of
infrastructure technology and management. But the engagement of scientists and
engineers in software and platform archictecture needs to be recongnized as a
key contribution to on-going and future dynamics of Science and Innovation in
Societies.

One key issue in this dynamic is the function of software as key assets for new
front of knowledge. This point has been and -still is- a matter of debates,
such as how to apply the FAIR (Findable, Accessible, Interoperable and
Reusable) principles to software \cite{noauthor_fair_nodate}, and proposals
such as the Citation File Format (CFF) \cite{chue_hong_software_2019}, a YAML
format for describing software metadata, or CodeMeta
\cite{druskat_citation_2019}, a JSON-LD conceptual vocabulary for software
repositories interoperability.

Many of these proposals are built around debates centered on the software
citation principles \cite{smith_software_2016}, summarized below:

\begin{description}
  \item[Importance:] Software should be considered a legitimate and citable product of research.
  \item[Credit and attribution:] Software citations should facilitate giving scholarly credit and recognizing.
  \item[Unique identification:] A software citation should include a globally unique identification.
  \item[Persistence:] Unique identifiers and metadata describing the software should persist.
  \item[Accessibility:] Software citations should facilitate access to the software itself and to its metadata.
  \item[Specificity:] Software citations should facilitate identification of the specific version of software that was used.
\end{description}

Based on these principles and taking some of the more recent proposals on how
to cite software, here follows a proposal of how the CorText Manager as a on
line application or software should be cited in academic papers.

\section{Cite CorText Manager with Bib\LaTeX} \label{example-biblatex}

Within the general issue expose in section 1, this note intend to normalize the
format of how the CorText Manager should be cited when it is accounted as a
software being developed and delivered as a online application for scientific
work by others.

When cited the format of citation is to be clearly defined as shown hereunder
\cite{cortext_manager_v2} is displayed in the bibliographic references. This
example uses the data described in the file {\em biblatex-software.bib}.  This
format is the one recommended to cite CorText Manager in an appropriate way.

\printbibliography[type=software,title={\small References to Software development (example)}]

\subsection{Bib\LaTeX \ file}

Below you see the content of the Bib\LaTeX \ file
\href{https://github.com/cortext/how-to-cite-cortext/blob/main/biblatex-software.bib}{\em biblatex-software.bib}
with the Bib\LaTeX \ Software extension\cite{softwareheritageorg_citing_2020}.

\VerbatimInput[fontsize=\small]{biblatex-software.bib}

When using this file you can see the result of reference format citing CorText
Manager \cite{cortext_manager_v2} on the references section of this document.

\section{Cite CorText Manager with Bib\TeX} \label{example-bibtex}

\printbibliography[type=misc,keyword=cortext,title={\small References (example)}]

\subsection{Bib\TeX \ file}

Below you see the content of the Bib\TeX \ file
\href{https://github.com/cortext/how-to-cite-cortext/blob/main/bibtex.bib}{\em bibtex.bib},
this file is created from
\href{https://github.com/cortext/how-to-cite-cortext/blob/main/CITATION.cff}{\em CITATION.cff}
converted by {\em cffconvert} tool.

\VerbatimInput[fontsize=\small]{bibtex.bib}

When using this file you can see the result of reference format citing CorText
Manager \cite{cortext_manager_v2_bibtex} on the references section of this document.

\section{CorText Manager metadata}

The CorText Manager metadata for academic citation is being maintained mainly in two files:

\begin{description}
  \item[CITATION.cff] Citation File Format (CFF) file with metadata about the software CorText Manager,
    this file is the main file to centralize all metadata required for citation.
  \item[biblatex-software.bib] BibLaTeX file with Software extension,
    this file is a transcription of all informations from the {\em CITATION.cff} file.
\end{description}

The example on Section \ref{example-biblatex} was generated by using the file
{\em biblatex-software.bib}, besides those two files there are some other
files that must be described:

\begin{description}
  \item[codemeta.json] CodeMeta file generated from {\em CITATION.cff} by the
    {\em cffconvert} tool, this file is a machine-readable file in a
    interchangeable JSON-LD format.
  \item[bibtex.bib] BibTeX file generated from {\em CITATION.cff} by the {\em
    cffconvert} tool, this file is useful for whom is using BibTeX as referencing
    system.
\end{description}

The example on Section \ref{example-bibtex} was generated by using the {\em
bibtex.bib} file, instructions on how to update those files can be found at
\href{https://github.com/cortext/how-to-cite-cortext/blob/main/AUTHORS.md}{\em AUTHORS.md}
document.

\printbibliography

\end{document}
