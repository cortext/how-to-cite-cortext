\documentclass{article}
\usepackage{graphicx}
\usepackage{hyperref}
\usepackage{fancyvrb}
\usepackage[backend=biber,datamodel=software]{biblatex}
\usepackage{software-biblatex}

\title{
  How To Cite CorText Manager
}

\author{
  Joenio Marques da Costa \\
  Université Gustave Eiffel
}

\date{\today}

\addbibresource{how-to-cite-cortext.bib}
\addbibresource{biblatex-software.bib}
\addbibresource{bibtex.bib}
\ExecuteBibliographyOptions{
  halid=true,
  swhid=true,
  swlabels=true,
  vcs=true,
  license=true}

\begin{document}

\maketitle

\section{Introduction}

Software plays a fundamental role in modern science in all fields of knowledge,
however it is not yet recognized as a first class product, something necessary
to generate academic credit and encourage researchers and organizations to
build software for science  \cite{alliez_attributing_2020}.

This panorama has been the subject of debates, such as how to apply the FAIR
(Findable, Accessible, Interoperable and Reusable) principles to software
\cite{noauthor_fair_nodate}, and proposals such as the Citation File Format
(CFF) \cite{chue_hong_software_2019}, a YAML format for describing software
metadata, or CodeMeta \cite{druskat_citation_2019}, a JSON-LD conceptual
vocabulary for software repositories interoperability.

Many of these proposals are built around debates centered on the software
citation principles \cite{smith_software_2016}, summarized below:

\begin{description}
  \item[Importance:] Software should be considered a legitimate and citable product of research.
  \item[Credit and attribution:] Software citations should facilitate giving scholarly credit and recognizing.
  \item[Unique identification:] A software citation should include a globally unique identification.
  \item[Persistence:] Unique identifiers and metadata describing the software should persist.
  \item[Accessibility:] Software citations should facilitate access to the software itself and to its metadata.
  \item[Specificity:] Software citations should facilitate identification of the specific version of software that was used.
\end{description}

Based on these principles and taking some of the more recent proposals on how
to cite software, here follows a proposal of how the CorText Manager software
should be cited in academic papers.

\section{Cite CorText Manager with Bib\LaTeX} \label{example-biblatex}

Below is an example of how the CorText Manager when cited
\cite{cortext_manager_v2} is displayed in the bibliographic references, this
example uses the data described in the file {\em biblatex-software.bib} and it
is the recommended way to cite CorText Manager.

\printbibliography[type=software,title={\small References (example)}]

\subsection{Bib\LaTeX \ file}

Below you see the content of the Bib\LaTeX \ file
\href{https://github.com/cortext/how-to-cite-cortext/blob/main/biblatex-software.bib}{\em biblatex-software.bib}
with the Bib\LaTeX \ Software extension\cite{softwareheritageorg_citing_2020}.

\VerbatimInput[fontsize=\small]{biblatex-software.bib}

When using this file you can see the result of reference format citing CorText
Manager \cite{cortext_manager_v2} on the references section of this document.

\section{Cite CorText Manager with Bib\TeX} \label{example-bibtex}

\printbibliography[type=misc,keyword=cortext,title={\small References (example)}]

\subsection{Bib\TeX \ file}

Below you see the content of the Bib\TeX \ file
\href{https://github.com/cortext/how-to-cite-cortext/blob/main/bibtex.bib}{\em bibtex.bib},
this file is created from
\href{https://github.com/cortext/how-to-cite-cortext/blob/main/CITATION.cff}{\em CITATION.cff}
converted by {\em cffconvert} tool.

\VerbatimInput[fontsize=\small]{bibtex.bib}

When using this file you can see the result of reference format citing CorText
Manager \cite{cortext_manager_v2_bibtex} on the references section of this document.

\section{CorText Manager metadata}

The CorText Manager metadata for academic citation is being maintained mainly in two files:

\begin{description}
  \item[CITATION.cff] Citation File Format (CFF) file with metadata about the software CorText Manager,
    this file is the main file to centralize all metadata required for citation.
  \item[biblatex-software.bib] BibLaTeX file with Software extension,
    this file is a transcription of all informations from the {\em CITATION.cff} file.
\end{description}

The example on Section \ref{example-biblatex} was generated by using the file
{\em biblatex-software.bib}, besides those two files there are some other
files that must be described:

\begin{description}
  \item[codemeta.json] CodeMeta file generated from {\em CITATION.cff} by the
    {\em cffconvert} tool, this file is a machine-readable file in a
    interchangeable JSON-LD format.
  \item[bibtex.bib] BibTeX file generated from {\em CITATION.cff} by the {\em
    cffconvert} tool, this file is useful for whom is using BibTeX as referencing
    system.
\end{description}

The example on Section \ref{example-bibtex} was generated by using the {\em
bibtex.bib} file, instructions on how to update those files can be found at
\href{https://github.com/cortext/how-to-cite-cortext/blob/main/AUTHORS.md}{\em AUTHORS.md}
document.

\printbibliography

\end{document}
