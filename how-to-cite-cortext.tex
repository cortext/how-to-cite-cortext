\documentclass{article}
\usepackage{graphicx}
\usepackage{hyperref}

\title{
  How To Cite CorText Manager
}

\author{
  Joenio Marques da Costa \\
  Université Gustave Eiffel
}

\date{\today}

\begin{document}

\maketitle

\section{Introduction} % motivation

Guia simplificado passo a passo para apoiar autores de software a fazer o software ser FAIR.
Os principios FAIR surgiram de conceitos do area de gerenciamento de dados, o acronimo significa
Findable, Accessible, Interoperable e Reusable.
Sao uma bandeira para promover boas praticas em gerenciamento de dados, mas nao sao aplicavei diretamente a software
devido a natureza diferente dos dois objetos, dados e software sao de longe dois artefatoss muito diferentes, apesar de
em certas circustancias serem tratados iguais.

Ajuda a publicar em repositorios publicos e controle de versao.
Indica selecionar uma licenca e da apoio e informacao de como escolher.
Indica registrar o codigo numa indice registro de alguma comunidade,
indica diversos servicos, por pais, por dominio, por plataforma, etc.
Crie as condicoes para o software ser citavel.
Este eh o ponto do meu interesse para definir como Cortext deve ser citado,
aqui parece ser a referencia mais fresca e atualizada,
a recomendacao eh usar CodeMeta e Citation File Format CFF,
era o q eu ja vinha sentindo q estes eram mesmo as ultimas recomendacoes sobre o tema.
E por fim recomenda boas praticas de qualidade, testes, documentacao etc,
nao se aplica para o Cortext pois o cortext ja esta estabelecido nesse contexto
\cite{noauthor_fair_nodate}.



Ha uma infinidade de ferramentas para ler, gerar ou converter CFF de e from para outros formatos, etc.
Esta pagina tem recomendacoes como usar CITATION.cff file, como gerar DOI e o que fazer qdo o software eh proprietario,
neste caso n eh possivel gerar doi pois nao se ha o repositorio do projeto, entao ha recomendacoes de como
citar e como dar instrucoes para que outros citem o projeto, exatamente o que necessito para definir corretamente
para o Cortext, devo ler com atencao e tomar notas etalhadas deste texto aqui neste documento.
TODO ler e tomar notas
\cite{saragon_standard_2017}.


CodeMeta deve ser gerar um arquivo codemeta.json e colocar na raiz do repositorio Git, Github, ok mas quais as
diferencas, similaridades, usar um ou outro em relacao ao arquivo CITATION.cff? Qual usar?
Este artigo tem todas as respostas, devo ler urgente e com atencao, eh o ponto fundamental de como definir o arquivo e como
se relacionado com codemeta etc...
nesse link eh possivel encontrar as versoes e links para doc de cada especificacao https://citation-file-format.github.io/about/
\cite{druskat_citation_2019}.


Este documento oferece um resumido, generico checklist para desenvolvedores de software
(seja software livre ou proprietario) usado em pesquisas podem usar para garantir que eles
estao seguindo boas praticas sobre citacao de software. Isto ira ajudar desenvolvedores de
obter credito para o software que eles criam, e melhorara transparencia, reproducibilidade
e reuso.
Quais sao os formatos de metadados apropriados?
O formato CodeMeta.json [3]  captura metadados de software, e eh entendido como um crescente
numero de repositorio digital, eh tambem facil de converter este para muitos esquemas
comuns de metadados usado por repostorios de software.
O Citation File Format (CFF) [4] eh um arquivo no formato YAML 1.2 legivel para humanos e
maquinas que prove metadata para citacao de software.
\cite{chue_hong_software_2019}.

\section{Conclusions}

* criar um espaco no site www.cortext.net e documentar como citar, com links e referencias, exemplos, etc
* OK criar arquivo `CITATION.cff` com metadados do CorText, autores, abstract, url, etc
  * https://citation-file-format.github.io/cff-initializer-javascript cliente web para gerar arquivo inicial
* OK gerar arquivo `codemeta.json` a partir do `CITATION.cff`, outros formatos sao suportados tambem
  * https://github.com/citation-file-format/cff-converter-python
  * `cffconvert -i CITATION.cff -f codemeta -o codemeta.json`
* publicar `CITATION.cff` e `codemeta.json`

\bibliographystyle{plain}
\bibliography{how-to-cite-cortext}

\end{document}
